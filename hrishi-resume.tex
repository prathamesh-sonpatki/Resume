%%%%%%
% Hrishikesh K B - Resume
% Edited using Kile
% Uses LaTeX and moderncv
%%%%%%

\documentclass[11pt,a4paper]{moderncv}
\moderncvtheme[blue]{classic}

\usepackage[utf8]{inputenc} % character encoding
\usepackage[scale=0.8]{geometry} % adjust the page margins

\AtBeginDocument{\setlength{\maketitlenamewidth}{10cm}}
\AtBeginDocument{\recomputelengths}                     % required when changes are made to page layout lengths

% personal data
\firstname{Hrishikesh}
\familyname{K B}
\title{Resume}
\address{2/4 NGO Quarters\\Kalpetta North,Wayanad}{Kerala, India}
\mobile{+91-9995613762}
\email{hrishi.kb@gmail.com}
\homepage{stultus.in}
\extrainfo{IRC Nick: stultus}
% \photo[64pt]{picture}                         % '64pt' is the height the picture must be resized to and 'picture' is the name of the picture file;

% --------------------------------------
% CONTENT
% --------------------------------------

\begin{document}
\maketitle

\section{Education}
\cvlistitem{Pursuing a Bachelors degree in \textit{Computer Science and Engineering} from College of Engineering Munnar.}

\section{Technical skills}
\cvcomputer
{\textbf{Operating Systems}}{Linux, Windows}
{\textbf{Programming Languages}}{Python , C , C++ , Java , PHP , HTML/CSS, JavaScript , Ajax}
\cvcomputer
{\textbf{Development Tools}}{ vim , git , bash , Eclipse}
{\textbf{Frameworks}}{ Django , Wordpress , jQuery }
\cvcomputer
{\textbf{Software Packages}}{MediaWiki,MySQL,SQLite,PostgreSQL,LaTeX,Gimp,\\Adobe Photoshop}
{}{}

% --------------------------------------
% WORK EX
% --------------------------------------

\section{Work Experience and Positions}

\cventry{May 2011}{Kerala State Legislative Assembly Election}{Media Programmer}{Government of Kerala}{}{Implemented an improved version of the previously developed application to display the result for public and media personals in Devilkulam division of Idukki District.}
\cventry{October 2010}{Election to the Local Self-government Institutions in Kerala}{Media Programmer}{Government of Kerala}{}{Implemented an application to display the result for public and media personals in Devilkulam division of Idukki District.} \\
\cventry{July 2009 -- April 2010}{Placement Cell}{Student Coordinator}{College of Engineering Munnar}{}{}


\subsection{Open Source Contributions}

\cventry{June 2010 -- present}{Swathanthra Malayalam Computing}{Contributor,Wiki Administrator}{}{http://smc.org.in}{ A Community of Free/Libre
software developers who works for Localization, Standardization and Development of Free/Lire
Software applications in Malayalam. Currently project has 40+ developers and 200+ nondeveloper members}

\cventry{February 2011 -- present}{SILPA}{Contributor}{}{http://silpa.org.in}{Silpa, Swathanthra Indian Language Processing Applications is a platform for porting existing and upcoming language processing applications to the web. Silpa can also be used as a python library or as a web-service from other applications.}

\cventry{February 2011 -- present}{Chamba Project}{Contributor,Wiki Administrator}{}{http://chambaproject.in/}{Chamba project is an ambitious effort to create a Swathanthra (Free/Libre/Open/Mukt) \\Animation Movie by pooling in contributions from people around the world and funding \\artists directly.}

% --------------------------------------
% PROJECTS
% --------------------------------------

%\pagebreak
\section{Projects}

\cvlistitem
{\textbf{Ml-ReCaptcha}  
  \\A Malayalam Captcha generator module written in Python. \\
    Which generates captcha from scanned books and helps to  \\
    digitalize books.Submitted as the 6th semester project (Minor Project) \\
}

\cvlistitem
{\textbf{Shingling} -- {\small https://github.com/stultus/silpa/tree/master/src/silpa/modules/shingling}
  \\Module for SILPA (Swathanthra Indian Language Processing Applications)\\
    written in python which generates set of unique "shingles"—contiguous  \\
    subsequences of tokens in a document—that can be used to gauge the similarity of two documents.\\
}

\cvlistitem
{\textbf{Malayalam-Inscript-Transliteration} -- {\small https://github.com/stultus/Malayalam-Inscript-}
  \\Widget written in javascript that can be integrated with MediaWiki \\
    and other websites which helps to type Malayalam using Inscript keyboard layout \\
}

\cvlistitem
{\textbf{3in1-Malayalam-Transliteration} -- {\small https://github.com/stultus/3in1-Mal-Input}
  \\Widget written in javascript that can be integrated with MediaWiki \\
    and other websites which helps to type Malayalam using Three different\\
    input methods including Inscript and two other phonetic keyboard layouts
}




% --------------------------------------
% TALKS ETC.
% --------------------------------------

\section{Talks given and Workshops conducted}
\cventry{September 2010}{Introduction to Blogging}{SKMJ High School Kalpetta}{}{}
{Presented an introductory talk on Blogging accompanied with a hands-on workshop on designing and  publishing weblog on popular blogging services for about 35 High School students.}
\cventry{Feb 2010}{Web 2.0}{Aarush10 , College of Engineering Munnar}{}{}
{Presented an introductory talk on web 2.0} 

% --------------------------------------
% Events Conducted
% --------------------------------------

\section{Events Organized}

\cventry{October 2010}{Wikipedia Academy Calicut}{with St. Joseph's College Devagiri,Calicut}{}{}
{Half day public outreach event to coach academics and other experts in how to contribute to Wikipedia, and to foster a positive impression of Wikipedia.\\
 \url{http://ml.wikipedia.org/wiki/wp:Calicut_wikipedia_Academy_1}}

\cventry{Feb 2010}{Aarush10}{}{}{}
{Internal technical festival of Association for Computer Science and Engineering Students-ACSES, College of Engineering Munnar.}

% --------------------------------------
% INTERESTS HOBBIES
% --------------------------------------

\section{Interests and Hobbies}

\cvlistitem{An avid reader on technology,fiction and non-fiction books and websites}
\cvlistitem{Active Wikipedian at Malayalam and Sanksrit Wikimedia Projects
 \\http://ml.wikipedia.org/wiki/user:Hrishikesh.kb
  \\http://sa.wikipedia.org/wiki/user:Hrishikesh.kb}
% \cvlistitem{I watch a lot of movies and listen to Hindi and Malayalam film songs} 
% --------------------------------------
\end{document}