%%%%%%
% Hrishikesh K B - Resume
% Edited using Kile
% Uses LaTeX and moderncv
%%%%%%

\documentclass[11pt,a4paper]{moderncv}
\moderncvtheme[blue]{classic}

\usepackage[utf8]{inputenc} % character encoding
\usepackage[scale=0.8]{geometry} % adjust the page margins

\AtBeginDocument{\setlength{\maketitlenamewidth}{10cm}}
\AtBeginDocument{\recomputelengths}                     % required when changes are made to page layout lengths

% personal data
\firstname{Hrishikesh}
\familyname{K B}
\title{Resume}
\address{2/4 NGO Quarters\\Kalpetta North,Wayanad}{Kerala, India}
\mobile{+91-9995613762}
\email{hrishi.kb@gmail.com}
\homepage{stultus.in}
\extrainfo{IRC Nick: stultus}
%\photo[54pt]{img.jpeg}                         % '64pt' is the height the picture must be resized to and 'picture' is the name of the picture file;

% --------------------------------------
% CONTENT
% --------------------------------------

\begin{document}
\maketitle

\section{Education}
\cvlistitem{Pursuing a Bachelors degree in \textit{Computer Science and Engineering} from College of Engineering Munnar.}

\section{Technical skills}
\cvcomputer
{\textbf{Operating Systems}}{Gnu/Linux, Windows}
{\textbf{Programming Languages}}{Python , C , C++ , Java(Preparing) , PHP , HTML/CSS, JavaScript , Ajax}
\cvcomputer
{\textbf{Development Tools}}{VIM, Git, Mercurial, bash , Eclipse, Omnet++}
{\textbf{Frameworks}}{ Django , Wordpress , Mixim}
\cvcomputer
{\textbf{Software Packages}}{MediaWiki,MySQL,SQLite,PostgreSQL,\\LaTeX,Gimp,Adobe Photoshop}
{}{}

% --------------------------------------
% WORK EX
% --------------------------------------

\section{Work Experience and Positions}

\cventry{May 2011}{Kerala State Legislative Assembly Election}{Media Programmer}{Government of Kerala}{}{Implemented an improved version of the previously developed application to display the result for public and media personals in Devilkulam division of Idukki District.}
\cventry{October 2010}{Election to the Local Self-government Institutions in Kerala}{Media Programmer}{Government of Kerala}{}{Implemented an application to display the result for public and media personals in Devilkulam division of Idukki District.} \\
\cventry{July 2010 -- January 2012}{Placement Cell}{Student Coordinator}{College of Engineering Munnar}{}{}


\subsection{Open Source Contributions}

\cventry{}{S.M.C}{Contributor,Wiki Administrator}{}{http://smc.org.in}{ Swathanthra Malayalam Computing,a community of Free/Libre
software developers who works for localization, standardization and development of Free/Lire
software applications in malayalam. Currently project has 40+ developers and 200+ nondeveloper members}

\cventry{}{SILPA}{Contributor}{}{http://silpa.org.in}{Silpa, Swathanthra Indian Language Processing Applications is a platform for porting existing and upcoming language processing applications to the web. Silpa can also be used as a python library or as a web-service from other applications.}

\cventry{}{Chamba Project}{Contributor,Wiki Administrator}{}{http://chambaproject.in/}{Chamba project is an ambitious effort to create a Swathanthra (Free/Libre/Open/Mukt) \\Animation Movie by pooling in contributions from people around the world and funding \\artists directly.}
\cventry{}{Diaspune}{Wiki Administrator}{}{http://wiki.diaspune.net/}{Diaspune is Diaspora Community in Pune. Diaspune promotes Diaspora Decentralised Social Network by organising meetups and creating artworks. We also encourage more people to contribute to diaspora software.}
%\cventry{}{Askbot}{Contributor}{}{https://github.com/stultus/askbot-devel/}{ASKBOT is a StackOverflow-like Q\&A forum, written in python - Django}


% --------------------------------------
% PROJECTS
% --------------------------------------

%\pagebreak
\section{Projects}

\cvlistitem
{\textbf{Geographic Routing in Wireless Multimedia Sensor Networks}  
  \\A C++ project to simulate the geographic routing in wireless multimedia \\
    sensor networks using Mixim framework and Omnet++ - an opensource simulation engine.  \\
    This is an ongoing project which has to be submitted as the 8th semester\\
   project(Major Project) \\
}

\cvlistitem
{\textbf{Ml-ReCaptcha}  
  \\A Malayalam Captcha generator module written in python,which generates captcha\\
    from scanned books and helps todigitalize books.\\
    Submitted as the 6th semester project (Minor Project) \\
}

\cvlistitem
{\textbf{Shingling} -- {\small https://github.com/stultus/silpa/tree/master/src/silpa/modules/shingling}
  \\Module for SILPA (Swathanthra Indian Language Processing Applications)\\
    written in python which generates set of unique "shingles"—contiguous  \\
    subsequences of tokens in a document—that can be used to gauge the similarity of two documents.\\
}

\cvlistitem
{\textbf{3in1-Malayalam-Transliteration} -- {\small https://github.com/stultus/3in1-Mal-Input}
  \\Widget written in javascript that can be integrated with MediaWiki \\
    and other websites which helps to type Malayalam using Three different\\
    input methods including Inscript and two other phonetic keyboard layouts
}




% --------------------------------------
% TALKS ETC.
% --------------------------------------

\section{Talks given and Workshops conducted}
\cventry{Sept 2010}{Introduction to Blogging}{SKMJ High School Kalpetta}{}{}
{Presented an introductory talk on Blogging accompanied with a hands-on workshop on designing and  publishing weblogs on popular blogging services for about 35 High School students.}
\cventry{Feb 2010}{Web 2.0}{Aarush10 , College of Engineering Munnar}{}{}
{Presented an introductory talk on web 2.0} 
\cventry{Aug 2011}{Distributed Social Networking}{ACSES , College of Engineering Munnar}{}{}
{Presented an introductory talk on Distributed Social Networking} 
\cventry{Sept 2011}{GIT - Distributed Version Control System}{College of Engineering Munnar}{}{}
{Presented a seminar on Git - A Distributed Version Control System as part of Academic Seminar Presentation}
\cventry{Dec 2011}{Contributing to FOSS Projects}{Astitva-11,College of Engineering Munnar}{}{}
{Presented an introductory talk on Contributing to Free and Open Source Software Projects}  
% --------------------------------------
% Events Conducted
% --------------------------------------

\section{Events Organized}

\cventry{October 2010}{Wikipedia Academy Calicut}{with St. Joseph's College Devagiri,Calicut}{}{}
{Half day public outreach event to coach academics and other experts in how to contribute to Wikipedia, and to foster a positive impression of Wikipedia.\\
 \url{http://ml.wikipedia.org/wiki/wp:Calicut_wikipedia_Academy_1}}

\cventry{February 2010}{Aarush10}{}{}{}
{Internal technical festival of Association for Computer Science and Engineering Students-ACSES, College of Engineering Munnar.}

\cventry{December 2012}{Astitva-12}{}{}{}
{Internal technical festival of College of Engineering Munnar.}

% --------------------------------------
% INTERESTS HOBBIES
% --------------------------------------

\section{Interests and Hobbies}

\cvlistitem{An avid reader on technology,fiction and non-fiction books and websites\\}
\cvlistitem{Active Wikipedian at Malayalam Wikimedia Projects
 \\http://ml.wikipedia.org/wiki/user:Hrishikesh.kb
  \\http://ml.wikisource.org/wiki/user:Hrishikesh.kb\\}
\cvlistitem{Active community member and novice contributor of Diaspora Project - The open source, privacy aware , distributed social network project.\\}
\cvlistitem{Hobbyist digital designer\\}
  
% \cvlistitem{I watch a lot of movies and listen to Hindi and Malayalam film songs} 
% --------------------------------------
%---------------- Declaration-------------------
\section{Declaration}
\cvlistitem{I hereby declare that the information furnished above is true to the best of my knowledge and belief.\\ \\ \\}

\begin{minipage}{0.5\textwidth}
\begin{flushleft}
  \large
\bfseries Munnar
\end{flushleft}
\end{minipage}
\begin{minipage}{0.5\textwidth}

\begin{flushright}
  \large
\bfseries Hrishikesh \textsc{K B}
\end{flushright}
\end{minipage}

%------------------------------------------


\end{document}